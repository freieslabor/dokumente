\documentclass[a4paper,12pt]{scrartcl}
\usepackage[utf8]{inputenc}
\usepackage[T1]{fontenc}
\usepackage[ngerman]{babel}
\usepackage[legal]{labdoc}

\title{Satzung des Freies~Labor~e.~V.}
\date{05.~Juni 2014}

\begin{document}
\maketitle


\section{Name, Sitz, Geschäftsjahr}
\begin{enumerate}
  \item Der Verein trägt den Namen Freies~Labor~e.~V.
  \item Er hat seinen Sitz in Hildesheim und wird dort in das Vereinsregister
    eingetragen.
  \item Das Geschäftsjahr ist das Kalenderjahr.
\end{enumerate}

\section{Zweck}
\begin{enumerate}
  \item Der Verein ist parteipolitisch und weltanschaulich neutral.
  \item Der Verein setzt sich zum Zweck:
    \begin{itemize}
      \item
         die Förderung der Erziehung und Volksbildung, insbesondere der
         Informatik- und Medienkompetenz der breiten Öffentlichkeit, sowie
         Aufklärung über und kritische Betrachtung von Risiken und
         Möglichkeiten neuer Technologien
      \item Kunst und Kultur in Hinblick auf den schöpferischen Umgang mit
        Technologie zu fördern
    \end{itemize}
  \item Der Vereinszweck soll insbesondere verwirklicht werden durch:
    \begin{itemize}
      \item die Bereitstellung und Pflege einer Räumlichkeit sowie der zur
        Verwirklichung der Vereinszwecke nötigen Infrastruktur
      \item die Organisation und Durchführung von lokalen Zusammenkünften und
        Informationsveranstaltungen sowie Öffentlichkeitsarbeit
      \item die Zusammenarbeit und der Austausch mit nationalen und
        internationalen Gruppierungen, deren Ziele mit denen des Vereins
        vereinbar sind
    \end{itemize}
\end{enumerate}

\section{Selbstlosigkeit und Gemeinnützigkeit}
\begin{enumerate}
  \item Der Verein ist selbstlos tätig; er verfolgt ausschließlich und
    unmittelbar gemeinnützige Zwecke im Sinne des Abschnitts "`steuerbegünstigte
    Zwecke"' der Abgabenordnung und ist nicht auf eigenwirtschaftliche Zwecke
    ausgerichtet.
  \item Mittel der Körperschaft dürfen nur für die satzungsmäßigen Zwecke
    verwendet werden.
  \item Die Mitglieder erhalten in ihrer Eigenschaft als Mitglieder keine
    Zuwendungen aus Mitteln der Körperschaft.
  \item Es darf keine Person durch Ausgaben, die dem Zweck der Körperschaft
    fremd sind, oder durch unverhältnismäßig hohe Vergütungen begünstigt werden.
\end{enumerate}

\section{Mitgliedschaft}
\begin{enumerate}
  \item Jede natürliche oder juristische Person kann Mitglied des Vereins
    werden. Bei Minderjährigen ist die Zustimmung der gesetzlichen Vertreterin/
    des gesetzlichen Vertreters erforderlich.
  \item Die Mitgliedschaft im Verein ist auf zwei Arten möglich:
    \begin{itemize}
      \item Ordentliche Mitglieder gestalten das Vereinsleben durch ihre aktive
        Teilnahme mit. Sie besitzen eine Stimmberechtigung auf den
        Mitgliederversammlungen des Vereins.
      \item Fördermitglieder unterstützen den Verein vorrangig durch ihren
        regelmäßigen finanziellen Beitrag. Sie besitzen keine Stimmberechtigung
        auf den Mitgliederversammlungen.
    \end{itemize}
  \item Die Beitrittserklärung erfolgt in Textform gegenüber dem Vorstand,
    dieser entscheidet auch über den Antrag.
  \item Die Mitgliedschaft endet durch Austrittserklärung, durch Ausschluss,
    durch Tod von natürlichen Personen oder durch Auflösung und Erlöschung von
    juristischen Personen.
  \item Ein Austritt ist jederzeit möglich und wird durch Willenserklärung in
    Textform gegenüber dem Vorstand vollzogen.
  \item Wenn ein Mitglied gegen die Ziele und Interessen des Vereins schwer
    verstoßen hat oder trotz Mahnung mit dem Beitrag für 3~Monate im Rückstand
    bleibt, so kann es durch den Vorstand mit sofortiger Wirkung ausgeschlossen
    werden. Dem Mitglied muss vor der Beschlussfassung Gelegenheit zur
    Stellungnahme gegeben werden. Der Vorstand muss dem auszuschließenden
    Mitglied den Beschluss in Textform unter Angabe von Gründen mitteilen. 
  \item Gegen den Ausschließungsbeschluss oder die Nichtaufnahme kann innerhalb 
    einer Frist von 21~Tagen nach Mitteilung des Ausschlusses Berufung eingelegt
    werden, über den die nächste Mitgliederversammlung entscheidet. Bis zum 
    Beschluss der Mitgliederversammlung ruht die Mitgliedschaft.
\end{enumerate}

\section{Beiträge}
\begin{enumerate}
  \item Es werden Mitgliedsbeiträge erhoben.
  \item Details regelt eine Beitragsordnung, die die Mitgliederversammlung
    beschließt.
\end{enumerate}

\section{Organe des Vereins}
\begin{enumerate}
  \item Organe des Vereins sind
  \begin{itemize}
    \item die Mitgliederversammlung
    \item der Vorstand
  \end{itemize}
\end{enumerate}

\section{Mitgliederversammlung}
\begin{enumerate}
  \item Die Mitgliederversammlung ist mindestens einmal jährlich einzuberufen.
  \item Eine Mitgliederversammlung ist außerdem einzuberufen, wenn es das
    Vereinsinteresse erfordert, oder wenn die Einberufung von mindestens 15\%
    der Vereinsmitglieder in Textform und unter Angabe des Zweckes und der
    Gründe verlangt wird.
  \item Die Einberufung der Mitgliederversammlung erfolgt in Textform durch den
    Vorstand unter Wahrung einer Einladungsfrist von mindestens 2~Wochen bei
    gleichzeitiger Bekanntgabe einer vorläufigen Tagesordnung. Die Frist beginnt
    mit dem Versanddatum. Das Einladungsschreiben gilt dem Mitglied als
    zugegangen, wenn es an die letzte vom Mitglied des Vereins in Textform
    bekannt gegebene Adresse gerichtet ist.
  \item Die Mitgliederversammlung als das oberste beschlussfassende Vereinsorgan
    ist grundsätzlich für alle Aufgaben zuständig, sofern bestimmte Aufgaben
    gemäß dieser Satzung nicht einem anderen Vereinsorgan übertragen wurden.
    Ihr sind insbesondere die Jahresrechnung und der Jahresbericht zur
    Beschlussfassung über die Genehmigung und die Entlastung des Vorstandes
    schriftlich vorzulegen. Sie bestellt zwei Rechnungsprüfer, die weder dem
    Vorstand noch einem vom Vorstand berufenen Gremium angehören und auch nicht
    Angestellte des Vereins sein dürfen, um die Buchführung einschließlich
    Jahresabschluss zu prüfen und über das Ergebnis vor der
    Mitgliederversammlung zu berichten.\\
    Die Mitgliederversammlung entscheidet z.~B. auch über
    \begin{itemize}
      \item Aufgaben des Vereins
      \item An- und Verkauf sowie Belastung von Grundbesitz
      \item Beteiligung an Gesellschaften
      \item Aufnahme von Darlehen
      \item Beschluss der Beitragsordnung
      \item Satzungsänderungen
      \item Auflösung des Vereins
    \end{itemize}
  \item Die Mitgliederversammlung gibt sich bei Bedarf eine Geschäftsordnung.
  \item Jede satzungsmäßig einberufene Mitgliederversammlung wird als
    beschlussfähig anerkannt, sofern mindestens 25\% der Mitglieder anwesend
    sind. Falls dieser geforderte Anteil nicht erreicht wird, ist die darauf
    folgende Mitgliederversammlung unabhängig von der Anzahl der erschienen
    Mitglieder beschlussfähig. Auf diesen Umstand muss in der Einladung zur
    Mitgliederversammlung besonders hingewiesen werden. Jedes ordentliche
    Mitglied hat eine Stimme. Fördermitglieder sind berechtigt, an den
    Versammlungen ohne Stimmrecht teilzunehmen.
  \item Die Mitgliederversammlung fasst ihre Beschlüsse mit einfacher Mehrheit
    der anwesenden Mitglieder, sofern in dieser Satzung nicht anders geregelt.
    Bei Stimmengleichheit gilt ein Antrag als abgelehnt.
  \item Die Ausübung des Stimmrechts auf der Mitgliederversammlung ist nur 
    möglich, wenn bis zum Zeitpunkt der Inanspruchnahme des e.~g. Rechts
    alle offenen Mitgliedsbeiträge des entsprechenden Mitglieds beglichen
    wurden.
\end{enumerate}

\section{Der Vorstand}
\begin{enumerate}
  \item Der Vorstand besteht aus 3~Mitgliedern: der/dem Vorstandsvorsitzenden,
    der/dem stellvertretenden Vorsitzenden und der Schatzmeisterin/dem
    Schatzmeister. Er vertritt den Verein gerichtlich und außergerichtlich. 2
    der 3 Vorstandsmitglieder sind vertretungsberechtigt.
  \item Der Vorstand wird von der Mitgliederversammlung für die Dauer von einem
    Jahr gewählt. Die Bestätigung des Vorstandes oder die Wiederwahl der
    Vorstandsmitglieder ist möglich. Die jeweils amtierenden Vorstandsmitglieder
    bleiben im Amt, bis Nachfolger gewählt sind.
  \item Dem Vorstand obliegt die Führung der laufenden Geschäfte des Vereins. Er
    hat insbesondere folgende Rechte:
    \begin{itemize}
      \item Gremien benennen und ihnen Mittel und Zuständigkeiten zuweisen
    \end{itemize}
  \item Vorstandssitzungen finden mindestens vierteljährlich statt.
  \item Mitglieder sind grundsätzlich berechtigt, an Vorstandssitzungen ohne
    Stimmrecht und ohne Rederecht teilzunehmen. Der Vorstand kann für einzelne
    Tagesordnungspunkte beschließen, diese unter Ausschluss der restlichen
    Mitglieder zu behandeln. Der Grund für den Ausschluss der Mitglieder muss im
    Protokoll festgehalten werden.
  \item Die Einladung zu Vorstandssitzungen erfolgt durch ein Mitglied des
    Vorstands in Textform unter Einhaltung einer Einladungsfrist von mindestens
    7~Tagen. Die Einladung muss außerdem an geeigneter Stelle für alle Mitglieder
    des Vereins veröffentlicht werden.
  \item Vorstandssitzungen sind beschlussfähig, wenn mindestens zwei Drittel der
    Mitglieder des Vorstandes anwesend sind. Der Vorstand fasst seine Beschlüsse
    mit einfacher Mehrheit der Vorstandsmitglieder.
  \item Dringende Beschlüsse des Vorstands können auch in Textform oder
    fernmündlich im Umlauf gefasst werden. Der Umlaufbeschluss 
    wird abgebrochen, sobald ein Vorstandsmitglied der Dringlichkeit
    widerspricht. Dringende Beschlüsse müssen auf der nächstfolgenden regulären
    Vorstandssitzung bestätigt werden. 
  \item Ist die Anzahl der Vorstandsmitglieder z.~B. durch Rücktritt auf unter 3
    gesunken, ist der restliche Vorstand verpflichtet, unverzüglich, spätestens
    jedoch innerhalb von 14~Tagen, zu einer zeitnahen Mitgliederversammlung
    einzuladen.
  \item Für vakant gewordene Vorstandsposten wird auf der nächsten
    Mitgliederversammlung jeweils eine Nachfolgerin/ein Nachfolger bestimmt, 
    die/der für die restliche Dauer der Amtszeit seiner Vorgängerin/seines
    Vorgängers im Amt bleibt.
  \item Der Vorstand gibt sich eine Geschäftsordnung, worin unter anderem die
    Aufgabenteilung des Vorstandes geregelt wird.
  \item Der Vorstand soll sich an den Beschlüssen der regelmäßigen 
    Plenumssitzungender Mitglieder orientieren.
\end{enumerate}

\section{Satzungsänderung}
\begin{enumerate}
  \item Für Satzungsänderungen ist eine Dreiviertel-Mehrheit der anwesenden
    Vereinsmitglieder erforderlich. Über Satzungsänderungen kann in der
    Mitgliederversammlung nur abgestimmt werden, wenn auf diesen
    Tagesordnungspunkt bereits in der Einladung zur Mitgliederversammlung
    hingewiesen wurde und der Einladung sowohl der bisherige als auch der
    vorgesehene neue Satzungstext beigefügt wurde.
  \item Zur Änderung des Vereinszwecks ist die Zweidrittel-Mehrheit aller
    Vereinsmitglieder erforderlich, wobei die Zustimmung der nicht anwesenden
    Mitglieder per Textform erfolgen kann.
  \item Satzungsänderungen, die von Aufsichts-, Gerichts- oder Finanzbehörden
    aus formalen Gründen verlangt werden, kann der Vorstand von sich aus
    vornehmen. Diese Satzungsänderungen müssen allen Vereinsmitgliedern
    alsbald per Textform mitgeteilt werden.
\end{enumerate}

\section{Beurkundung von Beschlüssen}
\begin{enumerate}
  \item Die in Vorstandssitzungen und in Mitgliederversammlungen gefassten
    Beschlüsse sind schriftlich niederzulegen und vom Vorstand, sowie von
    der Protokollantin/dem Protokollanten, die/der vor jeder Sitzung bestimmt 
    wird, zu unterzeichnen.
\end{enumerate}

\section{Auflösung des Vereins und Vermögensbindung}
\begin{enumerate}
  \item Für den Beschluss, den Verein aufzulösen, ist eine Dreiviertel-Mehrheit
    der in der Mitgliederversammlung anwesenden Mitglieder erforderlich. Der
    Beschluss kann nur nach rechtzeitiger Ankündigung in der Einladung zur
    Mitgliederversammlung gefasst werden.
  \item Bei Auflösung oder Aufhebung des Vereins, des Verlustes seiner
    Rechtsfähigkeit oder bei Wegfall seines steuerbegünstigten Zweckes fällt
    das Vermögen des Vereins nach Erfüllung sämtlicher Verpflichtungen an
    eine durch die letzte Mitgliederversammlung bestimmte steuerbegünstigte
    Körperschaft, die es unmittelbar und ausschließlich für gemeinnützige
    Zwecke im Sinne des §2 zu verwenden hat.
\end{enumerate}



\end{document}
% vim: set tw=80 et sw=2 ts=2:
